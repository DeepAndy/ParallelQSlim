\chapter{Conclutions}
\thispagestyle{empty}% no page number in chapter title page

The goal of this work was to create a parallel algorithm which would simplify planar surfaces and at the same time preserve high level of details in areas with complex geometry. The results show that the goal was achieved. We can clearly see that complex shapes like plants on desks are almost entariely preserved, whereas, walls or floors are descirbed with a few big triangles. Appendix A shows the results where the best approximation is for quardic error metric with all vertex's attributies. However, at the same time, this metric is the slowest one, because of the number of parameters to jointly optimize for every edge.

Summarizing, the algorithm is able to generate high quality progressive meshes in reasonable time, which was one of the most importatnt aspects. The results are promising and in some cases are better than commercially available products. The time of processing is biggest flaw of the approach. We can reduce this problem by increasing aggressiveness, however, the quality and our assumptions will suffer from it. Despite the fact that parallelization gives us in the first iteration almost 4 times speedup, all approximations have to be generated beforehand for the streaming purposes.
